\chapter{Antecedentes}


\section{Marco Teórico}
% +-- TODO --*
% - Figura: 
%	- Kai
% 	- Hapao

\subsection{Profiling}
\par La técnica de \emph{profiling} consiste en medir ciertos aspectos de interés en la ejecución del software, como por ejemplo: Tiempo de ejecución de un método, número de llamados de un método, cantidad de memoria utilizada por las instancias de una clase, número de distintas instancias de una clase, entre otros. 
\par El análisis apoyado por profilers se le denomina \emph{análisis dinámico} ya que los datos son obtenidos en tiempo de ejecución. Es decir, se analiza lo que realmente sucede durante la ejecución del software, lo que otorga información valiosa para prever el comportamiento del sistema en producción.
\par Posteriormente el usuario de los datos puede generar reportes, gráficos y/o visualizaciones que presentan un diagnóstico veraz del sistema analizado. 
\par Los productos de Object Profile  entregan reportes visuales usando Roassal. Tanto Kai como Hapao entregan visualizaciones interactivas (ver Figuras \ref{fig:kai} y \ref{fig:hapao}) que facilitan la comprensión del sistema y la detección de anomalías para una rápida intervención en el código.

\subsection{SUnit framework}
% Explicar test method, unit tests


\section{Trabajo Relacionado}
% +-- TODO --*
% - Figura: mostrar visualizaciones en herramientas
\subsection{Revisión de la literatura}
\par La mayoría de la investigación y aplicaciones relacionados a los tests va por el lado del coverage, que es uno de los indicadores más importantes al momento de determinar la confiabilidad de un software. 
\par En este aspecto algunos ejemplos 

\subsection{Revisión de herramientas de cobertura de test existentes}
