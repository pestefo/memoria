\chapter{Introducción} 
%% Intro
\par 
Dado la naturaleza de este tipo de código, su estudio e investigación ha intentado principalmente mejorar su efectividad para hacer software cada vez más confiables. Sin embargo, el problema diseño y mantención del código de test lleva bastante tiempo sin ser enfrentado en forma correcta y con la profundidad necesaria. 


%% Terminologia

%% Motivacion

% - Contexto: Práctica de Testing, Unit Testing
% Los test juegan un papel clave en la confiabilidad de 
% - Relevancia del Problema: Mantenibilidad

% - Alternativas de la Solución

% - Descripción de la solución (muy general)

%% Objetivos
\section{Objetivos}
\par El objetivo general de este trabajo es crear una herramienta que permita a los desarrolladores refactorizar y reestructurar sus test de una manera más fácil y mejorar la performance de estos.

\subsection*{Objetivos Específicos}
\begin{itemize}
\item Identificación de las métricas relevantes para caracterización de tests methods desde el punto de vista de un análisis dinámico
\item Desarrollar una visualización que permita detectar redundancia y solapamiento entre grupos de test methods (o unit tests)
\item Investigar refactorizaciones automáticas y semi-automáticas para unit tests y sus implicancias en performance y cobertura
\item Desarrollar una interfaz gráfica efectiva y usable para el uso cotidiano dentro del desarrollo de software
\end{itemize}

%% Organizacion del documento