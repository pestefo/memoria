\chapter{Implementación}\chaplabel{implementacion}

%=======
\section{Arquitectura}\seclabel{impl-arq}

\par El proyecto TestSurgeon fue completamente implementado en el lenguaje Pharo Smalltalk\footnote{Pharo Project [en línea] \textless\url{http://www.pharoproject.org}\textgreater  [Consulta: 12/08/2013] } ya que es en lenguaje en el que está escrito el framework de profiling utilizado (componente principal del proyecto). Pharo es un lenguaje de programación que implementa fielmente el paradigma de orientación a objetos. Sus características principales son, primero: entorno de desarrollo robusto totalmente integrado al lenguaje, y segundo: el sistema sigue una orientación a objetos pura, es decir, todo en Pharo es un objeto: Strings, Números, Clases, Métodos, Collecciones, Paquetes, etc. 

\par Para ejecutar Pharo, es necesario utilizar una máquina virtual la cual es específica de cada sistema operativo y actúa como interfaz de este último. La imagen de Pharo es una ``fotografía'' (o {\it snapshot}) del sistema Pharo congelado en el tiempo, específicamente de la última vez que se ejecutó. Contiene todas las instancias creadas y en el estado en que estaban justo antes de que la máquina virtual se cerrara.

\par Sobre Pharo están: \emph{Spy}, \emph{Roassal} y \emph{Spec}, las principales dependencias de TestSurgeon. \emph{Spy} es un framework de profiling~\cite{Berg11h} usado en TestSurgeon, se comentará en detalle sobre éste en la \secref{impl-spy}. \emph{Roassal} es un software que permite desarrollar visualizaciones interactivas en forma fácil y rápida. Es usado por TestSurgeon para implementar la \emph{Test Difference blueprint} (ver \secref{test-diff-blueprint}), en la \secref{impl-viz} se hablará de esta. Finalmente \emph{Spec} es un framework para construir interfaces gráficas y fue usado para la implementación del navegador de tests, ver \secref{impl-browser}.

\fig{H}{0.7}{impl/arq.pdf}{Arquitectura de \emph{TestSurgeon}}


%=======
\section{Profiling}\seclabel{impl-spy}
 
\par La componente principal de TestSurgeon es la que realiza el profiling  de la ejecución de los unit tests. Por eso se escogió una herramienta que fuera fácil de utilizar y que permitiera obtener la mayor cantidad de datos sobre la ejecución de un test.
\par El framework de profiling escogido fue \emph{Spy} y está escrito en el lenguaje Pharo. Entre sus características está que es un profiler orientado a los métodos. Considerando que en Pharo todas las interacciones corresponden a objetos que se comunican a través de mensajes, este punto es muy importante. Si se quiere conocer el comportamiento de un test con gran detalle, el hacer profiling de los métodos que son llamados durante su ejecución es fundamental. 

\par Además, Spy posee una arquitectura muy simple y fácil de extender. Consta de cuatro clases principales: {\tt Profiler}, {\tt PackageSpy}, {\tt ClassSpy} y {\tt MethodSpy}. {\tt Profiler} es la clase principal con las características necesarias para la instrumentación, ejecución y obtención de datos. Las demás clases contienen información sobre el profiling de los paquetes instrumentados y de sus clases y métodos respectivamente. 

\par La clase {\tt MethodSpy} técnicamente es un \emph{wrapper} de un método en Pharo (una instancia de la clase {\tt CompiledMethod}) que acumula datos sobre el contexto de su ejecución. Para esto, la clase provee los métodos {\tt beforeRun: with: in:} y {\tt aftefRun: with: in: } que son ejecutados justo antes e inmediatamente después de ejecutado el método instrumentado. Así se obtienen los datos del impacto de la ejecución de dicho método en el objeto y en los objetos con los cuales colabora. Estos métodos son abstractos y deben concretarse en las aplicaciones que decidan usar Spy extendiendo sus clases. 

%== Figura: (UML) Clases y extensión
\fig{H}{1}{impl/spy-testsurgeon.pdf}{Diagrama de clases de {\tt Spy-Core} y {\tt TestSurgeon-Core-Spy}}

\par Para hacer uso de Spy se deben entonces extender las clases antes mencionadas y acondicionarlas al dominio específico del problema. Estas clases son: {\tt TSProfiler}, {\tt TSPackage}, {\tt TSClass}, {\tt TSMethod}. Para las aplicaciones que usan esta información, se necesitan obtener datos de ejecución tanto de los tests como del código testeado. Es por esto que para representar un método de test se tiene también la clase {\tt TSTestMethod} que encapsula una instancia de {\tt TSMethod} correspondiente a un test method.

\par En el resto de la sección se describe en detalle las clases que componen al paquete {\tt TestSurgeon-Core-Spy}.

%  ---* TSProfiler *---
\subsection{La clase {\tt TSProfiler}}

\par Es la clase principal. Su interfaz componen los siguientes constructores {\tt buildForClassCategory:} y {\tt buildForPackagesMatching:}. En ambos, el argumento es un objeto String, en el primero especifica una categoría y en el segundo una expresión regular que representa un grupo de categorías o paquetes de software de Pharo. Luego de obtenida la categoría o las categorías, las instrumenta creando las clases {\tt TSPackage}, {\tt TSClass} y {\tt TSMethod}, posteriormente corre los tests y finalmente retorna la instancia del profiler.

\par La instancia de {\tt TSProfiler} tiene los siguientes métodos:

\begin{table}[h] 
    \centering 
    \begin{tabular}{|l|l|}
    	\hline
\textbf{Selector} & \textbf{Resultado} \\ \hline \hline
{\tt allTestMethods}	& retorna una colección de todos los tests ejecutados (instancias\\
						&  de {\tt TSMethod}  \\ \hline
{\tt allTestedMethods} & retorna una colección de todos los métodos testeados (instancias  \\
						& de {\tt TSMethod} \\ \hline
{\tt compiledToSpyMethod:} & encapsula un método compilado (instancia de {\tt CompiledMethod}) \\
						& retornándolo como instancia de {\tt TSMethod} . \\ \hline
{\tt plainClasses} & retorna una colección con todas las clases instrumentadas (instancias de  \\ 
						&{\tt TSClass}) que no corresponden a unit tests. \\ \hline
{\tt testClasses} & retorna una colección con todas las clases instrumentadas (instancias de \\ 
						&{\tt TSClass}) que corresponden a un unit test. \\ \hline
    \end{tabular}
    \caption{API pública de las instancias de {\tt TSProfiler}}
    \tablabel{impl-tsprofiler}
\end{table} 


%  ---* TSPackage & TSClass *---
\subsection{Las clases {\tt TSPackage} y {\tt TSClass}}

\par Estas dos clases, representan a los paquetes y clases instrumentadas respectivamente. En general no se necesitó información particular de los paquetes por lo que no hay nada que comentar sobre la implementación de {\tt TSPackage}.

\par {\tt TSClass} representa tanto a las clases testeadas como a los unit tests, y posee mayoritariamente métodos de ayuda o \emph{helpers} para las distintas aplicaciones (visualización, clustering, etc) como por ejemplo: {\tt allMethodsTestedBy:} donde se le entrega un {\tt TSTestMethod} y se retorna todos los métodos definidos en su clase que son llamados durante la ejecución de ese test.  

%  ---* TSMethod  *---
\subsection{La clase {\tt TSMethod} }
\par El profiling de Spy es orientado a los métodos, por tanto en esta clase es donde se registran todos los datos de la ejecución para caracterizar a los tests y poder diferenciarlos. Cada instancia de {\tt TSMethod} representa un método ejecutado, es decir, tanto el test que comienza la ejecución como todos los métodos llamados durante esta. Los datos son almacenados en variables de instancia, y existe solo una instancia por método definido en el software a analizar, entonces, un método ejecutado tendrá la información sobre su ejecución en distintos tests donde haya sido llamado. Las variables de instancia se definen a continuación:

\begin{table}[h] 
    \centering 
    \begin{tabular}{|l|l|}
    	\hline
\textbf{Selector} & \textbf{Resultado} \\ \hline \hline
{\tt testedBy}	& Diccionario que asocia cada test con las veces que el \\
				& método fue llamado por éste durante la ejecución del primero\\ \hline
{\tt originalMethod} & instancia de {\tt CompiledMethod} correspondiente al  \\
						& método encapsulado. \\ \hline
{\tt receiverFootprintsMapping} & Diccionario que asocia un test con todos los objetos \\
						& \emph{receivers} del método actual durante su ejecución. \\ \hline
    \end{tabular}
    \caption{Atributos principales de {\tt TSMethod}}
    \tablabel{impl-tsmethod}
\end{table} 

\newpage
\par Su comportamiento está dividido en dos partes principalmente: los métodos de profiling donde se registran los datos de ejecución, y sus métodos accesores: métodos que entregan la información obtenida. Los métodos de profiling son dos: {\tt beforeRun: with: in:} y {\tt aftefRun: with: in:}, a continuación se muestra el código del primero. 

%== Codigo: \ref{code:before-run} y en \ref{code:after-run}
\begin{codeWithLineNumbers}
<i>beforeRun:</i> methodName <i>with:</i> listOfArguments <i>in:</i> receiver
	| testMethod |
	" Test en ejecucion "
	testMethod := self profiler class currentTestMethod.
	currentTestMethodSpy := 
			spyMethodMapping at: testMethod ifAbsentPut: 
				[ self profiler >> testMethod class name >> testMethod selector ].
		
	" Guardar en el diccionario de tests"	
	self addTestMethod: currentTestMethodSpy.

	" Guardar receiver "
	self updateReceiversWith: receiver for: currentTestMethodSpy.

\end{codeWithLineNumbers}

%\par Los dos métodos señalados son ejecutados por todos los métodos dentro de la ejecución del test, incluso este último. Sin embargo, la estrategia de obtención de los datos de ejecución es que los métodos instrumentados del código base sean los que almacenen los datos de esta. 

\par En el código de {\tt beforeRun:with:in:}, una instancia de {\tt TSMethod} lo recibe. Lo primero que realiza es obtener el test que originó la ejecución. Para esto, realiza un llamado al profiler para solicitar el test que se está ejecutando en ese momento. El test se agrega al diccionario {\tt spyMethodMapping} que representa una biyección entre los tests como {\tt CompiledMethod} y su método Spy que lo encapsula. 

\par Posteriormente, el test encapsulado se agrega como llave al diccionario {\tt testedBy} (diccionario de tests que cada método del código base posee y contabiliza el número de llamados durante cada test) y se incrementa en 1 el contador. Además, es necesario identificar el objeto recibidor del mensaje (objeto que ejecuta dicho método). Para esto cada método tiene un diccionario llamado {\tt receiverFootprintsMapping} donde se asocia el test con una colección de objetos que recibieron el mensaje durante su ejecución. Para esto, cada objeto va anotado por su valor de hash que lo representa en forma única e individual.

\par Las instancias de {\tt TSMethod} son entonces las principales fuentes de datos para la creación de métricas y visualizaciones. Para esto cuenta una API pública que se detalla a continuación:  

\begin{table}[h] 
    \centering 
    \begin{tabular}{|l|l|}
    	\hline
\textbf{Selector} & \textbf{Resultado} \\ \hline \hline

{\tt numberOfCallsFrom: } & Retorna el valor de la métrica \emph{Number of Calls}\\
						& (o {\tt NOC}, ver \secref{viz-metricas}). \\ \hline
{\tt numberOfReceiversDuringExecutionOf:} & Retorna el valor de la métrica \emph{Number of Different}  \\ 
						&\emph{Receivers} (o {\tt NODR}, ver \secref{viz-metricas}). \\ \hline
{\tt differencesBetweenNoCFrom:and:}	& Retorna la diferencia entre la métrica {\tt NOC} entre\\
						&  dos tests. \\ \hline
{\tt differencesBetweenNoDRFrom:and:} & Retorna la diferencia entre la métrica {\tt NODR}  entre\\
						&  dos tests. \\ \hline
{\tt numberOfTestMethodsThatCallIt} & Retorna el número de tests en los cuales es  \\ 
						& ejecutado.\\ \hline
{\tt unitTestThatTestsIt} & Retorna el número total de unit tests donde algún  \\ 
						& test haya lo ejecutado. \\ \hline						
    \end{tabular}
    \caption{API pública de las instancias de {\tt TSMethod}}
    \tablabel{impl-tsmethod2}
\end{table} 


%  ---* TSTestMethod  *---
\subsection{La clase {\tt TSTestMethod} }
\par Como se explicó anteriormente, las instancias de {\tt TSMethod} pueden encapsular un método compilado del código base o bien de los tests que inician la ejecución de éstos. La API pública presentada anteriormente está enfocada a consultas para métodos del código base. Para el análisis de los tests es necesario incluirlos también como entidades del modelo de clases. Para eso está {\tt TSTestMethod} que básicamente es un \emph{wrapper} de una instancia {\tt TSMethod} que encapsula un test method.

\par A continuación se presenta su API pública: 

\begin{table}[h] 
    \centering 
    \begin{tabular}{|l|l|}
    	\hline
\textbf{Selector} & \textbf{Resultado} \\ \hline \hline

{\tt numberOfTestedMethods } & Retorna el número de métodos testeados.\\ \hline
{\tt testedClasses} & Retorna una colección con las clases ({\tt TSClass}) que \\ 
						& contengan métodos testeados.\\ \hline
{\tt testedMethods}	& Retorna una colección con los métodos ({\tt TSMethod}) \\
						& testeados. \\ \hline
{\tt visualizeDifferencesWith:} & Retorna una instancia {\tt TSTestDifferenceBlueprint},\\
						&  la visualización comparativa con el test entregado\\
						&  como argumento. \\ \hline
{\tt compareWith:} & Retorna una instancia {\tt TSDynCoverageOtoOComp}, \\  
						&  un comparador 1-a-1  de test que provee métricas \\
						&  de similitud dinámica.\\ \hline		
{\tt staticCompareWith:} & Retorna una instancia {\tt TSStaticOtoOComp},  \\ 
						& un comparador 1-a-1 de test que provee métricas \\
						& de similitud estática.\\ \hline						
    \end{tabular}
    \caption{API pública de las instancias de {\tt TSTestMethod}}
    \tablabel{impl-tstestmethod}
\end{table} 



%=======
\section{Visualización}\seclabel{impl-viz}
%% Method Shape: hablar del ajuste logaritmico.
\par La implementación de \emph{Test Difference Blueprint} es la clase {\tt TSTestDifferenceBlueprint}. Para crear la visualización necesita de los datos de los tests a comparar, por lo cual dentro de sus atributos están {\tt redTest} y {\tt blueTest}, instancias de {\tt TSTestMethod}. Además, posee un comparador de similitud dinámica (instancia de {\tt TSDynCoverageOtoOComp}) el cual está almacenado en el atributo {\tt comparator}. Para la información general sobre toda la instrumentación y ejecución es que se necesita una instancia de {\tt TSProfiler} la cual está referenciada por el atributo {\tt profiler}. 

\par La visualización como tal corresponde instancia de {\tt ROView}, que es, en palabras simples, un contenedor de los elementos gráficos de la visualización y es el responsable de mostrarlos apropiadamente. Por otro lado, enviando el mensaje {\tt exportAs:}, el objeto crea un archivo de imagen vectorial (formato {\tt SVG}) con el {\tt String} entregado como parámetro como nombre. Técnicamente no hay más aspectos destacables que no se puedan apreciar en forma directa desde el código fuente. Más información sobre la visualización se puede encontrar en la \secref{test-diff-blueprint}. 

%=======
\section{Browser}\seclabel{impl-browser}
%http://hal.inria.fr/docs/00/70/80/67/PDF/SpecTechReport.pdf
\par El browser o navegador permite acceder rápidamente a los métodos a través de menúes que permiten seleccionar unit tests y los tests para realizar comparaciones rápidamente.  Esta interfaz está implementada completamente usando el framework Spec. Gráficamente consiste en cuatro espacios: Selector de Unit Tests, Selector de Test methods, Zona de Visualización y Zona de Código. La disposición de estas zonas es como se muestra en la \figref{caso-uso/2.png}. Para lanzar el browser se envía el mensaje {\tt openOn:} a {\tt TSBrowser} donde el argumento corresponde a una expresión regular que calza con el nombre de los paquetes a analizar. Por ejemplo, para analizar los tests de Spec, se debe ejecutar el comando {\tt TSBrowser openOn: 'Spec-*'} porque los paquetes de Spec siguen la regla de nombrarse con {\tt Spec-} como prefijo, y los paquetes instrumentados entonces serán: {\tt Spec-Core}, {\tt Spec-Examples}, {\tt Spec-Layout}, {\tt Spec-Tools}, {\tt Spec-Tools-Editor}, {\tt Spec-Widgets}, entre otros.  

\par Como las comparaciones entre ejecuciones de tests son 1-a-1, necesitamos primero escoger dos unit tests, uno para los candidatos a test rojo y otro para los candidatos azules. Cada uno de estos menúes de selección única son una instancia {\tt IconListModel} y muestran una lista de los unit tests ordenados por tamaño (cantidad de tests) de mayor a menor. Una vez seleccionados ambos unit tests, se despliegan los menúes inferiores, los selectores de test methods. Estos, también son objetos {\tt IconListModel} pero con algunas diferencias: el menú para escoger test rojo ordena los tests en orden alfabético; por su parte el menú para test azul inicialmente está en orden alfabético pero una vez se haya seleccionado un test azul se ordena según similitud dinámica descendente con respecto al test rojo. 

\par Cuando hay un test rojo y azul seleccionados, se refresca la visualización para mostrar la comparación de los elementos seleccionados y la zona de comparación de código. Sobre la última, para ahorrar tiempo, se reutilizó esta componente ({\tt DiffMorph}) desde el browser de administración y descarga de paquetes(\emph{Monticello Browser}\footnote{Dynamic Web Development with Seaside . Saving your Package to Monticello [en línea] \textless \url{http://book.seaside.st/book/getting-started/pharo/monticello} \textgreater [Consulta: 12/08/2013] }). El comparador de código, éste compara línea a línea el código y marca de color (rojo y verde, no sigue la convención de TestSurgeon) las diferencias entre estos. 
%=======
\section{Clustering y Comparación de tests}\seclabel{impl-clustering}

\par En la \secref{caso-estudio-clustering} se presenta el problema de la gran cantidad de posibles comparaciones que el desarrollador debe realizar usando el browser para detectar posibles \emph{test smells}. Es por eso que se realizó un experimento agrupando los tests del unit test {\tt ROMondrianViewBuilderTest} según los métodos que cubren sus test methods. Así crear clusters de tests similares y ahorrar una gran cantidad de comparaciones innecesarias.

\subsection{Adaptación del Algoritmo Agrupamiento Aglomerativo Jerárquico}\seclabel{impl-hac}

\par Para esto, se decidió utilizar el \emph{Algoritmo de Agrupamiento Aglomerativo Jerárquico}~\cite{defays1977efficient,sibson1973slink} (o \emph{Hierarchical Agglomerative Clustering}). Este algoritmo pretende crear una jerarquía de clusters a través de fusionar clusters más similares (o cercanos) en cada paso. Al comienzo, existen $N$ clusters de tamaño $1$, es decir. Luego de calcular todas las distancias entre estos se fusionan los clusters con menor distancia, y ahora existen $N-1$ clusters: $N-2$ de tamaño $1$ y el recién fusionado con $2$ elementos. El algoritmo sigue hasta que se tiene $1$ solo cluster de tamaño $N$. Las implementaciones comunes del algoritmo reciben de entrada un número $k$ que corresponde al número de clusters deseado, entonces, el algoritmo cuenta con una condición de término que corresponde a finalizar la ejecución cuando la última fusión realizada haya generado en total: $k$ clusters.

% Algoritmo original

\par Las restricciones del problema a resolver es obtener una cantidad ``razonable'' de clusters (ver \secref{exp-clustering-met}) que reduzcan la cantidad de comparaciones. Sin embargo, de poco sirve generar clusters muy heterogéneos ya que así se incluirían más casos de comparaciones innecesarias. Entonces no hay forma de determinar cuál $k$ es más apropiado para este caso. Para esto se adaptó el algoritmo modificando la condición de término transformándolo en el siguiente invariante: el algoritmo debe en cada iteración generar clusters con una similitud interna mayor o igual a $s_{min}$. Es decir, una vez que los clusters fusionados formen uno con similitud interna $s'$ tal que $s' < s_{min}$, dicho cluster debe deshacerse, y el algoritmo retorna la última colección de clusters que respetó el invariante. La similitud interna de un cluster se midió como el promedio de las similitudes de cobertura entre todos sus elementos, formalmente:

\[ \textrm{similitud\_interna}(C_k) = \dfrac{1}{\abs{C_k}\cdot(\abs{C_k}-1)} \sum_{i=0}^{\abs{C_k}} \sum_{j=0}^{\abs{C_k} - 1} g(C_{k_i},C_{k_j} )\]

\par Donde $g(C_{k_i},C_{k_j})$ es la medida de similitud por cobertura entre el test $t_i, t_j \in C_{k}$, definida en la \secref{exp1-metricas-sim}. La experiencia en el uso del browser indica que dos tests con similitud sobre 90\% son relevantes y vale la pena analizarlos. Entonces para descubrir la similitud interna mínima más apropiada se realizó un experimento ejecutando el algoritmo para distintos valores de esta métrica: 95\%, 90\%, 85\%, 80\% y 75\%. 

% Algoritmo adaptado (http://www.rosapolis.net/2008/04/21/escribir-algoritmos-en-latex/)

\par Otro aspecto relevante fue la elección de la métrica de similitud entre dos clusters. Entre las alternativas se decidió implementar tres distancias y evaluarlas en un experimento, estas son: 
\begin{description}
\item[\emph{Single Link}] La distancia entre dos clusters corresponde a la distancia entre los dos elementos (uno de cada cluster) más cercanos. En términos de la métrica de similitud por cobertura $g$, la distancia se define como: dados $C_p$ y $C_q$ clusters,
\[ D(C_p,C_q) = \max_{x \in C_p,\, y  \in C_q} g(x,y) \]

\item[\emph{Complete Link}] La distancia entre dos clusters corresponde a la distancia entre los dos elementos (uno de cada cluster) más lejanos. En términos de la métrica de similitud por cobertura $g$, la distancia se define como: dados $C_p$ y $C_q$ clusters,
\[ D(C_p,C_q) = \min_{x \in C_p,\, y  \in C_q} g(x,y) \]

\item[\emph{Average Link}] La distancia entre dos clusters corresponde al promedio total de distancias entre todos los elementos de ambos clusters.  En términos de la métrica de similitud por cobertura $g$, la distancia se define como: dados $C_p$ y $C_q$ clusters,
\[ D(C_p,C_q) = \dfrac{1}{\abs{C_p}\cdot\abs{C_q}} \sum_{x \in C_p} \sum_{y  \in C_q} g(x,y) \]


\end{description}

\fig{H}{0.7}{impl/single-complete-link.pdf}{Distancias mínima (Single Link) y máxima (Complete Link) entre dos clusters}

\par Finalmente el experimento de agrupamiento consistió en ejecutar el algoritmo de agrupación aglomerativo jerárquico para cada distancia y cada similitud interna mínima, con lo cual se obtuvo 15 particiones distintas de {\tt ROMondrianViewBuilderTest}. Luego de evaluados los resultados (ver \apref{exp-clustering}) se obtuvo que la mejor configuración corresponde a imponer una \textbf{similitud interna mínima} de \textbf{85\%} y usando la métrica de distancia \textbf{Complete Link}.


\subsection{Diagrama de Clases}

\par La implementación del algoritmo y el experimento descritos anteriormente consiste en 3 paquetes principales: {\tt TestSurgeon-Clustering},  {\tt TestSurgeon-Comparator} y {\tt TestSurgeon-Experiments}. 

\fig{H}{1}{impl/comparator.pdf}{Diagrama de clases del paquete {\tt TestSurgeon-Comparator}}
\fig{H}{1}{impl/clustering.pdf}{Diagrama de clases del paquete {\tt TestSurgeon-Clustering}}

\par El paquete {\tt TestSurgeon-Clustering} contiene las clases necesarias para agrupar tests. La clase {\tt TSHACDistance} define solamente el método abstracto {\tt distanceBetween:and:} que recibe como argumentos dos instancias de la clase {\tt TSTestCluster}, el cual debe ser sobrescrito por sus subclases. Estas son {\tt TSSingleLinkDistance}, {\tt TSCompleteLinkDistance} y {\tt TSAverageLinkDistance} los cuales implementan dicho método según la definición de distancia correspondiente. Por su parte la clase {\tt TSHierarchicalAgglomerativeClustering} contiene todo el comportamiento necesario para ejecutar el algoritmo salvo la métrica de distancia, que está representada por el método {\tt distanceMetric} y es sobrescrito por sus tres subclases: {\tt TSHACSingleLink}, {\tt TSHACCompleteLink} y {\tt TSHACAverageLink} los cuales retornan la clase que implementa dicha métrica. A continuación se muestra el código del método principal que ejecuta el algoritmo.

\newpage

\begin{codeWithLineNumbers}
<i>TSHierarchicalAgglomerativeClustering>>run</i>
run
	" Al comienzo hay N clusters con un solo elemento "
	clusters := elements collect: [:el | TSTestCluster new add: el. ].
	
	[
		"Invariante"
		(clusters collect: #internalSimilarity) min >= minInternalSimilarity.
	] 
		whileTrue: [ 
			"computar las similitudes entre clusters y ordenarlas en orden descendente "
			self updateSimilaritiesBetweenClusters.
			self groupClosestClusters
	].
		
	" el ciclo termino porque el ultimo merge produjo un cluster no deseado, por lo cual se va a separar"
	self reverseLastMerge. 
	
	" etiquetar clusters y retornarlos "
	self tagClusters.
	
	^ clusters
\end{codeWithLineNumbers}
 
\par La clase {\tt TSTestCluster} representa un grupo de test el cual responde a las siguientes mensajes: 

\begin{table}[h] 
    \centering 
    \begin{tabular}{|l|l|}
    	\hline
\textbf{Selector} & \textbf{Resultado} \\ \hline \hline

{\tt internalSimilarity } & Retorna el valor de su similitud interna.\\ \hline
{\tt mergeWith: anotherCluster} & Retorna una nueva instancia de {\tt TSTestCluster} \\ 
						& con la unión entre sus elementos y los de  \\ 
						& {\tt anotherCluster} que representa la fusión de \\
						& ambos. \\ \hline
{\tt similarityWith: anotherCluster}	& Retorna la similitud con {\tt anotherCluster} \\ 
{\tt using: aTSHACDistanceClass } &  usando la métrica de distancia entregada  \\
						& como argumento ({\tt aTSHACDistanceClass}).\\ \hline
{\tt includes: element} & Retorna verdadero si {\tt element} pertenece \\
						&  al cluster.\\ \hline
{\tt size:} & Retorna la cantidad de elementos contenidos. \\ \hline		
				
    \end{tabular}
    \caption{API pública de las instancias de {\tt TSTestCluster}}
    \tablabel{impl-tstestcluster}
\end{table}  

\vspace*{1cm}

\par El paquete {\tt TestSurgeon-Comparator} contiene a comparadores de test methods que ofrecen diversas métricas según el tipo de comparación que se realice. Las comparaciones pueden ser entre dos métodos de test ({\tt TSOneToOneComp}), entre un método de test y un grupo de tests ({\tt TSOneToManyComp}) y entre dos grupos de test ({\tt TSManyToManyComp}). Los dos últimos comparadores reutilizan el comparador uno-a-uno aplicándolo a colecciones de test, por lo cual detallaremos el primero. La clase {\tt TSOneToOneComp} es una clase abstracta que contiene dos atributos: {\tt redTest} y {\tt blueTest}, los tests a comparar. El comportamiento mínimo que cualquier extensión de esta clase debe implementar es de retornar el valor de similitud entre el test rojo y azul a través del método {\tt similarity}. Las dos clases que lo extienden son: {\tt TSDynCoverageOtoOComp} y {\tt TSStaticOtoOComp}. El primero implementa la métrica $g$ presentada anteriormente, y la segunda implementa la métrica $f$ (ver \secref{exp1-metricas-sim}) que mide la similitud de test según su código fuente. A continuación presentamos la API pública de cada una de estas clases.



\begin{table}[h] 
    \centering 
    \begin{tabular}{|l|l|}
    	\hline
\textbf{Selector} & \textbf{Resultado} \\ \hline \hline

{\tt similarity } & Retorna el valor de la métrica $g$ entre los \\
				 & tests rojo y azul. \\ \hline	
{\tt diffBetweenNoCallsOn:} & Retorna la diferencia entre la cantidad \\ 
{\tt  aTSMethod}		& de llamados recibidos por {\tt aTSMethod} durante  \\ 
						& la ejecución del test rojo y la del test azul.\\ \hline
{\tt diffBetweenNoDifferent-}	& Retorna la diferencia entre la cantidad \\ 
{\tt ReceiversOn: aTSMethod } &  de objetos distintos que recibieron el mensaje  \\
						& {\tt aTSMethod} durante la ejecución del test rojo \\
						& y la del test azul.\\ \hline
{\tt numberOfMethods- } & Retorna la cantidad de métodos cubiertos\\
{\tt TestedInCommon}						&  en común.\\ \hline
{\tt numberOfAllMethodsTested} & Retorna la cantidad de métodos cubiertos \\
						& en total por ambos tests. \\ \hline		
				
    \end{tabular}
    \caption{API pública de las instancias de {\tt TSDynCoverageOtoOComp}}
    \tablabel{impl-gcomp}
\end{table} 

\vspace*{1cm}

\begin{table}[h] 
    \centering 
    \begin{tabular}{|l|l|}
    	\hline
\textbf{Selector} & \textbf{Resultado} \\ \hline \hline

{\tt similarity} & Retorna el valor de la métrica $f$ entre los \\
				 & tests rojo y azul. \\ \hline	
{\tt nbOfCommonLines } & Retorna la cantidad de líneas de código \\
				 & en común entre ambos tests.\\ \hline
{\tt nbOfTotalLines} & Retorna el total de líneas de código en \\
				 & ambos tests (contados sin repetición e ignorando espacios)\\ \hline
{\tt minDifferenceToClone}	& Retorna el menor valor comparando la diferencia entre \\
				 & el número de lineas en común y el numero de líneas de código \\
				 & definido por cada test. Si el valor es cero, indica que el \\
				 & código de un test está completamente dentro del otro test.\\ \hline 
    \end{tabular}
    \caption{API pública de las instancias de {\tt TSStaticOtoOComp}}
    \tablabel{impl-fcomp}
\end{table} 

\clearpage
\par Finalmente, en el paquete {\tt TestSurgeon-Experiments}, la clase {\tt TSClusteringExp} realiza el experimento de correr el algoritmo de clustering bajo distintas tolerancias de similitud interna y distintas distancias de clustering cuyos resultados se pueden consultar en el \apref{exp-clustering}. El método principal de esta clase se presenta a continuación.

\begin{codeWithLineNumbers}
<i>TSClusteringExp>>run</i>
	| clusteringAlgorithm minIntSimilerities tests |
	" Obtener los tests a agrupar "
	tests := (profiler classAt: #ROMondrianViewBuilderTest) allTestMethodsAsTSTestMethods.
	" Definir las similitudes minimas a evaluar" 
	minIntSimilerities := #(0.95 0.9 0.85 0.8 0.75).
	
	" Correr el algoritmo con distintas distancias para cada similitud minima"
	TSHierarchicalAgglomerativeClustering subclasses do: [:HACAlgorithm |
		minIntSimilerities do: [:minIntSimilarity |
			|algo|
			" Inicializar el algoritmo y correrlo "
			algo := (HACAlgorithm 	
							elements: tests
			  				minimumInternalSimilarity: minIntSimilarity).
			algo run.
			
			" Adjuntar los resultados obtenidos "
			results addAll: algo results.
			clusteringAlgorithms add: algo.
		]
	].
	
	" Exportar los resultados a un archivo CSV "
	self exportResults.
\end{codeWithLineNumbers}


\section{Acceso al código}\seclabel{impl-repo}
\par Todo el código detallado en las secciones anteriores está almacenado en el sitio \emph{SmalltalkHub}\footnote{Smalltalk Hub [en línea] \textless\url{http://www.smalltalkhub.com}\textgreater [Consulta: 12/08/2013] }, en el repositorio ubicado en la siguiente url: 
\begin{center}
\url{http://smalltalkhub.com/\#!/~PabloEstefo/TestSurgeon/}
\end{center}