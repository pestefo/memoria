\chapter{Caso de estudio: Roassal}\chaplabel{caso-de-estudio}

\par En el siguiente capítulo se presentará una aplicación de \emph{TestSurgeon} analizando los tests de \emph{Roassal}. 

\section{Roassal: motor de visualización}

\par Roassal es un motor de visualización desarrollado en la plataforma Pharo. Roassal está hecho para visualizar e interactuar con datos arbitrarios definidos en términos de objetos y sus relaciones. Roassal es comúnmente utilizado para producir visualizaciones interactivas. El rango de aplicaciones usando Roassal es diverso. Por ejemplo, es muy utilizado en el análisis de software (ver \figref{roassal-example.png}), de hecho es utilizado en la visualización de TestSurgeon.

\par Roassal consta de 30 paquetes que contienen 301 clases que definen 4137 métodos que en total suman 29720 líneas de código. El correcto funcionamiento de este código es asegurado por X unit test que en total suman X test methods, cuya cobertura alcanza un 72.37 \%.

\fig{h}{0.35}{roassal-example.png}{Visualización hecha en Roassal: Jerarquía de la clase {\tt Collection} (arreglos, diccionarios y variantes) y sus clases hijas. El alto representa el número de métodos y el ancho el número de atributos}


% Hablar de Roassal
Evaluar escenarios
ROMondrianViewBuilderTest --> alto nivel