\begin{abstract}

% Problema
\par Actualmente la actividad de Testing es fundamental dentro del ciclo de desarrollo de cualquier proyecto de software serio. Es más, las metodologías ágiles elevan su relevancia dentro de la construcción del software a tal nivel que está prohibido añadir una nueva funcionalidad sin que se haya escrito previamente un test que la valide.

% Relevancia
\par A medida que el software crece en funcionalidades y cambian los requerimientos se vuelve más complejo. Es por eso que existen varias técnicas para restructurar el código para hacerlo más flexible a los cambios y permitir que crezca. 

\par Sin embargo, los test también crecen en número y en complejidad. Escenario en el cual abundan problemas de redundancia tanto en el código fuente de éstos como en su ejecución. Pero, a difernecia con el código ``funcional'', poco esfuerzo se ha realizado por parte de la industria y la academia para mantener una estructura y diseño limpio de los tests. 

\par Una de las consecuencias importantes de este problema es el gran tiempo que toma ejecutar todos los tests. Al tener tests redundantes, la ejecución tarda más tiempo del necesario lo hace que los desarrolladores disminuyan la frecuencia de la ejecución de estos, e inclusive minimicen la cobertura de sus tests. Esto último atenta críticamente en la confiabilidad del código funcional y por ende de la aplicación completa.

\par En este trabajo se propone una herramienta para detectar problemas de diseño de los tests o \emph{test smells}. \emph{TestSurgeon} aborda este problema desde distintos aspectos de análsis de un test, tales como: su código fuente y su ejecución. A través de una intuitiva intefaz, el desarrollador puede navegar sobre las pruebas unitarias y realizar comparaciones entre tests ayudado por métricas dedicadas que lo guían para encontrar casos interesantes de posible restructuración. Además provee una completa visualización que condensa armónicamente una serie de métricas que describen y a su vez diferencian la ejecución de dos tests. Esto permite realizar un análisis comparativo eficaz entre dos tests. De esta manera, TestSurgeon permite detectar diferencias semánticas entre tests y encontrar redundancias entre estos para una posterior refactorización.

\par Se presenta también en detalle una experiencia de la aplicación de \emph{TestSurgeon} sobre los tests de \emph{Roassal}, un motor de visualización ágil. Se detallan tres escenarios de refactorización y restructuración de test que la herramienta permite detectar.  
% Este problema es particularmente importante cuando se automatiza el proceso de liberación de código a través de integración contínua. Donde antes de liberar una nueva funcionalidad se ejecutan todos los tests para verificar que no se ha roto ninguna funcionalidad previamente liberada. Por tanto, el tener tests redundantes tiempo valioso se pierde y provoca disminución en la velocidad del equipo de desarrollo.  

% Approach
\par 

% Resultados

% Conclusiones




\end{abstract}