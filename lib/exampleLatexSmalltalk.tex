\documentclass{article}
\usepackage{xcolor}
\usepackage{listings}
\input{macros}

\begin{document}

This is a plain example:

\begin{code}
fuelExample
	| sourceArray loadedArray |
	sourceArray := Array 
		with: 'a string' 
		with: Transcript
		with: 42
		with: #foo.
	"Store to a file"
	FLSerializer serialize: sourceArray toFileNamed: 'example.FL'.
	"Load from the file"
	loadedArray := FLMaterializer materializeFromFileNamed: 'example.FL'.
	"Check that the materialized Transcript is the right singleton instance."
	loadedArray second show: loadedArray first; flush.
	self validateMaterializedGraph: loadedArray.
\end{code}

This is how you can define pieces of bold, italics and underline:

\begin{code}
<b>fuelExample</b>
	| sourceArray loadedArray |
	sourceArray := Array
		with: 'a string' 
		with: Transcript
		with: 42
		with: #foo.
	"Store to a file"
	<i>FLSerializer serialize: sourceArray toFileNamed: 'example.FL'.</i>
	"Load from the file"
	<u>loadedArray := FLMaterializer materializeFromFileNamed: 'example.FL'.</u>
	"Check that the materialized Transcript is the right singleton instance."
	loadedArray second show: loadedArray first; flush.
	self validateMaterializedGraph: loadedArray.
\end{code}

This is another environment if you want to show number lines:

\begin{codeWithLineNumbers}
<b>fuelExample</b>
	| sourceArray loadedArray |
	sourceArray := Array
		with: 'a string' 
		with: Transcript
		with: 42
		with: #foo.
	"Store to a file"
	<i>FLSerializer serialize: sourceArray toFileNamed: 'example.FL'.</i>
	"Load from the file"
	<u>loadedArray := FLMaterializer materializeFromFileNamed: 'example.FL'.</u>
	"Check that the materialized Transcript is the right singleton instance."
	loadedArray second show: loadedArray first; flush.
	self validateMaterializedGraph: loadedArray.
\end{codeWithLineNumbers}






\end{document}
